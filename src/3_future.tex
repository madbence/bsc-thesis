\section{Kitekintés: automatizálás}

A microservice architektúra legnagyobb hátránya, hogy az egyes komponensek
kézi menedzselése nagyon kényelmetlen, egy adott komponensszám felett
pedig gyakorlatilag lehetetlen.

A problémára megoldást jelent, ha az egyes komponenseket egy egységes
felületen keresztül tudjuk menedzselni, a deploy pedig teljesen automatikusan
zajlik, a komponensek konfigurációja pedig szintén automatikus.

A közös interfészre megoldást nyújtanak a különböző \emph{containerization}
technológiák, pl. a Docker\cite{Docker}, ami izolációs konténereket nyújt alkalmazásoknak,
az különböző konténerek pedig a Docker szoftveren keresztül egységesen
menedzselhetők

Az automatikus deployra több megoldás is létezik, többek között ilyen
a Python alapú Ansible\cite{Ansible}, ami távoli szerverek menedzselését teszi lehetővé
egyszerűen. Ilyen a megfelelő komponensek feltelepítése, frissítése,
konfigurálása.

A konfigurációmenedzsment teszi lehetővé, hogy az egyes komponensek
képesek legyenek automatikusan felismerni, hogy a számukra releváns egyéb
szolgáltatások milyen végpontokon keresztül érhetőek el, értesüljenek
az esetleges szolgáltatáskiesésekről, konfigurációs változásokról.
Ilyen szoftver pl. a Consul\cite{Consul}, ami egy Go alapú szoftver, elsősorban
\emph{service registry}-ként, illetve konfigurációs adatbázisként működik,
de képes akár az egyes komponensek működését is monitorozni.
