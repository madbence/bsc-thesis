%------------------------------------------------------------------------------
\chapter{Bevezető}
%------------------------------------------------------------------------------

\section{A megoldandó probléma}

A Twitter közösségi platform segítéségével a felhasználók rövid szöveges
(maximum 140 karakteres) tartalmakat tudnak megosztani egymással.
Az átlagos felhasználó valamilyen mobil eszközön használja a szolgáltatást.
\cite{About-Twitter}

Mivel a közölt tartalmak (\emph{tweetek}) nagyon rövidek, a felhasználók
sokkal többet is tweetelnek, mint mondjuk egy blog esetén.
A tipikus felhasználó csak az elmúlt néhány óra eseményeit tudja követni
(vagy még kevesebbet), hiszen egész egyszerűen az információ sokkal
gyorsabban áramlik, mint ahogy a felhasználó azt fel tudná dolgozni.

Pontosan emiatt nagyon nehéz a platformot információk hatékony, gyors
terjesztésére használni. Egy tweet nagyon gyorsan elveszik az információk
rengetegében, több tweet azonban negatívan hat azokra a felhasználókra,
akik folyamatosan követik az eseményeket.

Az általam írt szoftver ezt a problémát próbálja meg megoldani úgy,
hogy a felhasználó követő bázisának szokásait elemezve megmondja azt az
idő intervallumot, amikor a tweet a legtöbb emberhez eljuthat.
Mivel a tweetek nézettségét mérni nagyon limitáltan lehet,
a rendszer gyakorlatilag fekete dobozként viselkedik,
így azt az egyszerű hipotézist állítottam föl,
hogy a legideálisabb időpont az, amikor a legtöbb felhasználó online,
azaz épp a Twittert használja valamilyen felületen keresztül.
Ez szintén nem mérhető közvetlenül, így online felhasználónak az számít,
aki éppen tweetelt (pl. az elmúlt 15 percben).

Természetesen ez a heurisztika nem pontos, hiszen nem foglalkozik azokkal,
akik nem tweetelnek, vagy esetleg szeretik akár több napra visszamenőleg
átolvasni az falukat. Ezek olyan tényezők, amiket közvetlenül nem,
közvetve is csak nagyon körülményesen lehet mérni, ezeknek a szokásoknak
a figyelembevétele messze túlmutat a dolgozat keretein.

\section{Szükséges lépések a megoldáshoz}

A probléma ismeretében a következő részproblémákat határoztam meg,
amiket a programnak meg kell oldania:

\begin{itemize}
  \item A szoftvernek rendelkeznie kell egy webes felülettel, amin keresztül a
felhasználó azonosítása megtörténhet.
  \item A szükséges adatokat \emph{on-demand} jelleggel nem lehet lekérdezni,
mert hosszadalmas folyamat, így szükség van a háttérben futó feldolgozó
egységekre.
  \item Az elosztott működést koordinálni kell, az egyes komponenseknek
kommunikálniuk kell egymással, ezt az összekötő réteget is meg kell valósítani.
  \item A folyamatok monitorozása fontos a rendszer elemzéséhez, így
szükséges egy rugalmas naplózó rendszer integrálása.
\end{itemize}

\section{A dolgozat felépítése}

A dolgozat második fejezetében ismertetem a rendelkezésre álló technológiákat,
azok szerepét a megvalósításban, röviden ismertetem a főbb tulajdonságaikat,
illetve megindoklom, hogy miért döntöttem az adott technológiák mellett,
esetenként összehasonlítva más alternatívákkal őket.

A harmadik fejezetben ismertetem az alkalmazás egyes tervezési döntéseit,
a lehetséges alternatívákkal együtt. Kitérek a felmerülő technikai problémákra,
illetve bemutatom a konkrét szoftvert feladatban kitűzött
problémák megoldására.

Az utolsó fejezet néhány mérési eredményt mutat be a rendszer működéséről,
illetve kitekintés, hogy hogyan lehetne további funkciókkal kiegészíteni
a rendszert, milyen további előnyök származnának belőle.
