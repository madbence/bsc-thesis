%------------------------------------------------------------------------------
% Feladatkiiras (a tanszeken atveheto, kinyomtatott valtozat)
%------------------------------------------------------------------------------
\clearpage
\begin{center}
\large
\textbf{FELADATKIÍRÁS}\\
\end{center}

A közösségi média platformok rohamos terjedésével felmerült a kérdés
a felhasználók részéről, hogy a megosztott tartalmakat hogyan tudnák a lehető
leghatékonyabban eljuttatni a célközönségüknek.
A Twitter az egyik legnépszerűbb ilyen platform, a felhasználói pedig leginkább
olyan emberek, akik az őket érdeklő információkról valós időben, azonnal
szeretnének tudni.

A hallgató feladata egy olyan webes alkalmazás megírása Node.js környezetben,
ami a tartalmat megosztani kívánó felhasználó számára információt ad arról,
hogy mely időpontokban kell megszólítania a követőit, hogy a közölt tartalom
a lehető legtöbb felhasználóhoz eljusson.

A hallgató feladatának a következőkre kell kiterjednie:

\begin{itemize}
\item Mutassa be a felhasznált technológiákat.
\item Készítse el az alkalmazást JavaScript nyelven.
\item Tegye képessé a programot arra, hogy elosztott környezetben,
  hálózaton keresztül kooperálva egyszerre több példány is fusson belőle.
\item Az elkészült alkalmazás belső működését monitorozza, tegye mérhetővé.
\end{itemize}
