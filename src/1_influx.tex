\section{InfluxDB}

Az InfluxDB egy ún \emph{time-series} adatbázis, ami elsősorban időfüggő
metrikák hatékony tárolására és elemzésére alkalmas\cite{InfluxDB}.
Az adatokat lehetőség van különböző adatbázisokban perzisztálni,
többek között akár LevelDB-n is.

A szoftver lehetőséget nyújt a tárolt adatok komplex elemzésére, egy SQL-hez
hasonló lekérdező nyelv segítségével, illetve a lekérdezés eredménye grafikusan
is megjeleníthető, így egyszerű diagramok készítése beépítetten támogatott.

Pl. ha szeretnénk perc felbontású adatokat kapni az alkalmazás átlagos
válaszidejéről az elmúlt egy órából, az alábbi lekéréssel tehetjük meg:

\begin{verbatim}
select mean(response_time)
from app
group by time(1m)
where time > now() - 1h
\end{verbatim}

Lehetőség van ún. \emph{continous query} futtatására is, ami arra szolgál,
hogy a beérkező adatokat aggregálja, azaz tipikusan mintát vegyen belőle,
és egy másik adatsorba továbbítsa. Ez utóbbi adatsor sokkal kisebb méretű,
hiszen csak aggregált (alacsonyabb felbontású) adatok vannak benne,
azonban a nagyobb mennyiségű adatok elemzését igénylő lekérdezések sebességét
radikálisan megnöveli. Ha egy metrika másodperc felbontású (pl. aktuális
CPU terhelés), egy éves időtávlatban valószínűleg nincs szükség ilyen
pontos adatokra, elég az egy órás felbontás.
Később, a nagyobb időintervallumokat használó lekérdezések nem közvetlenül
a másodperc felbontású adatsoron hajtódnak végre, hanem csökkentett
felbontásún, így egy kevés pontatlanság árán a töredékére csökkentettük
a feldolgozási időt.

Lehetőség van az adatok programozott lekérésére is külső API-n keresztül
(HTTP JSON, illetve elérhető graphite-kompatibilis interfész is), így
tetszőleges külső szoftverrel integrálható (ilyen például a Grafana).

Az adatbázis atomi eseményeket fogad, ezekhez tetszőleges adatokat csatolhatunk,
számos környezethez létezik integrációja (többet között Node.JS-hez is).
Az elkészült szoftverben az \verb=influx= modult használom, ami kényelmes
API-t biztosít az adatbázishoz.

\begin{js}
  var influx = require('influx');
  var client = influx({
    host: host,
    port: 8086,
    username: user,
    password: pass,
    database: db,
  });

  var data = {
    time: new Date(),
    attr1: val1,
    attr2: val2,
  };
  client.writePoint(series, data, opts, callback);
\end{js}

A háttérben a \verb=writePoint= hívás egy HTTP POST kérésre fordul le,
amit az adatbázis a REST API-ján keresztül tud fogadni.
Lehetőség van egyszerre több adatpont bevitelére is, a \verb=writePoints=
metódus segítségével.
