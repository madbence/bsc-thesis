\section{Microservice architektúra}

A microservice architektúra szinte pontosan az előző ellentéte.
Az egyes szoftverkomponensek önállóan is futtatható alkalmazások,
a futtatáshoz nagyon kevés erőforrásra van csak szükség, a feladatuk
ellátásához azonban szükség van sok másik komponensre.

A komponensek közti kommunikáció valamilyen jól definiált, nyelvfüggetlen
interfészen keresztül zajlik (pl HTTP fölötti JSON kommunikáció, vagy
valamilyen alacsony szintú, TCP/UDP fölött definiált protokoll).

A megoldás előnye, hogy a skálázás sokkal rugalmasabb, elegendő csak a szűk
keresztmetszetet jelentő komponenseket skálázni, így adott esetben
jelentősen kevesebb erőforrás is elengendő a megfelelő szolgáltatási
szint biztosításához.

Az architektúra további előnye, hogy technológiafüggetlen, hiszen az egyes
komponensek teljesen függetlenek egymástól, csak a megfelelő interfészeket
kell biztosítaniuk egymásnak. Egy rétegelt architektúrában ez többnyire nem
megoldható.

Természetesen hátrányok is vannak, az egyik legnyilvánvalóbb, hogy az
üzemeltetés bonyolultabb, amennyiben nem megfelelő eszközöket használunk.
A sok különálló komponens egyenkénti deployolása, illetve konfigurálása
manuálisan nagyon nehézkes, de vannak eszközök, amikkel az egész folyamat
teljesen automatizálható.
