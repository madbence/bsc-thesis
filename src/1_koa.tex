\section{\texttt{koa}}

A node legelterjedtebb webes keretrendszere az \verb=express=\cite{Express}
framework, ami a beépített \verb=http= modulra (webszerver) épül,
és kibővíti azt kényelmes routing lehetőségekkel,
illetve ún. \emph{middleware}-ek használatával szabhatjuk testre a működését.

A \verb=koa=\cite{Koa} ennek a következő generációs megfelelője,
elődjéhez képest sokkal letisztultabb, elegánsabb interfészt nyújt,
szinte teljesen a fejlesztőre bíz minden fontos architekturális döntést,
az aszinkron működéshez a \verb=co= modult és generátorokat használ.

\subsection{Middleware-alapú architektúra}

A \verb=koa= esetében a middleware egy egyszerű generátor, aminek a feladata
az érkező kérés lekezelése, majd a vezérlés továbbadása a következő
middleware rétegnek, majd végül az érkező válasz további manipulálása.
Természetesen egyik lépés sem kötelező, minden logika implementálása a
fejlesztőre van bízva. Egy tipikus middleware (ez esetben egy naplózó)
felépítése:

\begin{js}
var koa = require('koa');
var app = koa();

app
  .use(function* log(next) {
    var start = Date.now();
    yield* next;
    console.log(
      'Request to %s took %d ms',
      this.url,
      Date.now() - start
    );
  });
\end{js}

Első lépés az alkalmazás létrehozása, majd az alkalmazáshoz hozzáadjuk az
általunk megírt middleware-t a \verb=use= metódussal.
A generátor egyetlen paramétert vár (\verb=next=), amivel a vezérlést
tovább tudjuk adni a soron következő rétegnek.
A példa middleware egyszerű naplózást valósít meg, a szabványos kimenetre
írja az érkező HTTP kéréseket, illetve a hozzájuk tartozó válaszidőt.
Elsőként eltároljuk az aktuális időbélyeget (``leszálló'' ág),
majd a \verb=yield* next= utasítással delegáljuk a következő generátornak a
vezérlést (a \verb=yield*= kulcsszó nem szakítja meg a generátor futását,
csak tovább delegálja azt), végül a ``felszálló'' ágon elvégzi a szükséges
naplózást (\verb=console.log= hívás).
A működésre az egyik legszemléletesebb analógia a böngészők DOM eseménykezelése,
ahol a \verb=yield* next= előtti rész az esemény \emph{capture} fázisának,
az azutáni rész pedig a \emph{bubble} fázisnak felel meg.

Természetesen léteznek ennél egyszerűbb middleware-ek, egy \emph{hello world}
alkalmazás szerkezete az alábbi:

\begin{js}
var koa = require('koa');
var app = koa();

app.use(function* hello() {
  this.body = 'Hello world!';
});
\end{js}

A \verb=this= jelenti a HTTP kérés kontextusát, a \verb=body= adattag pedig
a kérésre adott választ jelenti. Ez lehet egy egyszerű \verb=String=, de akár
JSON objektum is, ekkor a megfelelő HTTP válasz fejléc is beállítódik
(\verb=Content-Type: application/json=), az objektum pedig szerializálódik.
A válasz opcionálisan akár \verb=Stream= is lehet, ez akkor lehet hasznos,
ha pl. fájlokat szeretnénk kiszolgálni, és a teljes eredményt nem kívánjuk
(vagy nem is tudjuk) a memóriában tárolni, hanem folyamatosan szeretnénk
a kliensnek továbbítani.

\subsection{Általános struktúra}

A \verb=koa= alkalmazások maguk is middleware-ként viselkednek,
így azokat egyszerűen tudjuk egymásba ágyazni.
Az általános konvenció az, hogy egy fő alkalmazásba felvesszük a közösen
használt modulokat, majd azok alá felvesszük a megfelelő route-okon hallgató
önálló alkalmazásokat, így az elkészült szoftver laza csatolású, hiszen
a használt alkalmazások tetszőlegesen válogathatóak össze, azokat könnyen
fel lehet használni újra.

A \verb=koa= beépítetten csak egyetlen middleware-t (\verb=respond=) tartalmaz,
aminek a feladata pusztán a kéréshez tartozó kontextus inicializálása, illetve
egy 404-es hibakód visszaadása, ha egyik további middleware sem kezelte
le a kérést (nem állított be választ).

\subsection{Routing megoldások}

Természetesen a fejlesztő megírhatja a saját routing modulját,
azonban rendelkezésre állnak kész megoldások, amik ezt nagyban megkönnyítik.
Ezek közül az egyik a \verb=koa-route= modul, ami a megfelelő HTTP
metódusokra illeszkedő middleware generátorfüggvényeket gyárt az általunk
megadott kontrollerhez

\begin{js}
var koa = require('koa');
var route = require('koa-route');
var get = route.get;
var app = koa();

app
  .use(get('/users/:id', showUser));
\end{js}

A \verb=get= függvény paraméterként egy sztring mintát,
illetve egy middleware-t vár, és egy olyan middleware-t ad vissza,
ami a megfelelő mintára illeszkedő HTTP kéréseket továbbítja az átadott
kontrollernek (paraméterként pedig átadja az illesztett \verb=id= paramétert),
minden mást kérést átenged a soron következő middleware-nek.

A másik modul a \verb=koa-mount=, ezzel komplett \verb=koa= alkalmazásokat
tudunk URL-transzparens módon egymásba illeszteni, azaz az egyes
alrendszereknek nem kell tudnia teljes URL-ről.

\begin{js}
var mount = require('koa-mount');

users
  .use(get('/:id', showUser));

messages
  .use(get('/:id', showMessage));

app
  .use(mount('/users', users))
  .use(mount('/messages', messages));
\end{js}

A példában két \verb=koa= alrendszert (\verb=users= és \verb=messages=)
fogunk össze, melyek a \verb=/users= illetve a \verb=/messages= útvonalakon
hallgatnak. Az egyes alkalmazások nem tudnak (nem is tudhatnak) a
teljes elérési útról. A modult használva tetszőleges alkalmazások fűzhetőek
össze anélkül, hogy a moduljuk kódját módosítani kéne, így az eredmény
egy lazán csatolt, könnyen bővíthető architektúra.

\subsection{REST API}

A \emph{REST}\cite{REST} (Representational state transfer) egy konvenció a HTTP fölött
történő kommunikációra. A rendszerben erőforrásokat definiálunk
(pl. \emph{felhasználó}, \emph{üzenet}, stb.), ezekhez azonosítót rendelünk,
az erőforrásokat pedig ezen az URI-n (Uniform Resource Identifier) tudjuk
elérni, illetve manipulálni. Erre szolgálnak a különböző HTTP metódusok:

\begin{description}
  \item[GET] Adott erőforrás (vagy erőforrások) elérésére szolgál.
    Idempotens művelet, azaz a kimenete mindig ugyanaz, a meghívások számától
    függetlenül, és biztonságos is, az erőforrást nem módosítja, manipulálja.
  \item[POST] Új erőforrást hoz létre, nem idempotens.
  \item[DELETE] Adott erőforrást töröl, idempotens, de nem biztonságos, hiszen
    módosítja az erőforrást.
  \item[PUT] Az erőforrás \emph{teljes} leírását frissíti, idempotens, de nem
    biztonságos művelet.
  \item[PATCH] Az erőforrást \emph{részlegesen} frissíti, ugyanazon
    tulajdonságokkal rendelkezik, mint a PUT metódus.
\end{description}

Ezek a műveletek mind jól lefordíthatóak a rendszerben lévő modellek
metódushívásaira, így célszerűnek láttam, hogy ezeket a nagyon hasonló
műveleteket mind kiemeljem paraméterezhető middleware-kbe, amiket az alkalmazás
fejlesztése során többször is újra fel lehet használni.

Ilyen jellegű művelet a \emph{resolve}, ami a megadott (\verb=/resource/:id=)
alakú URI-ket képes feloldani valódi erőforrásokra:

\begin{js}
module.exports = function (name, resolver) {
  return function* resolve(id, next) {
    this[name] = yield* resolver(id);
    yield* next;
  };
};
\end{js}
