%------------------------------------------------------------------------------
% Abstract in hungarian
%------------------------------------------------------------------------------
\chapter*{Kivonat}\addcontentsline{toc}{chapter}{Kivonat}

A Twitter napjaink egyik legnépszerűbb kozösségi platformja, segítségével
egyszerűen tudunk taralmat megosztani másokkal. Azonban az üzenet gyakran
nem jut el a célközönségéhez, ha azt nem a megfelelő időben publikáljuk.

Erre a problémára nyújt megoldást az általam írt szoftver. Követőink
viselkedése alapján megbecsüli, hogy milyen időintervallumban célszerű
\emph{tweetelni}, ha potenciálisan a letöbb követőnkhöz szeretnénk eljuttatni
az üzenetünket.

Platformként a Node.JS-t választottam, mert pontosan az ilyen jellegű
feladatokban szerepel jól, hatékonyan képes skálázódni, beépített megoldások
segítik a hatékony elosztott működést.

A JavaScript napjaink egyik legnépszerűbb programozási nyelvve, számos
programozási paradigmát (objektum-orientált, funkcionális, stb.) támogat,
teljesítményben pedig már felveszi a versenyt az eddig egyeduralkodónak
gondolt natív társaival.

A legfőbb szempont a tervezés során az volt, hogy lazán csatolt, könnyű
komponensekből építsem meg a rendszert, hiszen a jól skálázható, eloszott
működéshez ez elengedhetetlen.

\vfill

\selectlanguage{english}

%------------------------------------------------------------------------------
% Abstract in english
%------------------------------------------------------------------------------
\chapter*{Abstract}\addcontentsline{toc}{chapter}{Abstract}

Twitter is one of the most popular social networks, people can
easily share content with others. This method of content sharing can be very
efficient, albeit it heavily depends on the time of the publishing.

My software addresses this problem by analysing the behavior of our follower
base. After this step, the application tells us when should we publish our
content to maximize the number of impressions.

My chosen platform was Node.JS, because it is designed for these kind of
applications. It scales well, one can write distributed network applications
very easily.

The JavaScript language is very popular, it supports various programming
paradigms (object-oriented, functional, etc.), its performance is very
impressive, it is comparable to native languages like C++, Java, C\#.

My top priority during the development was that the individual software
components should be loosely coupled and lightweight. These properties
are necessary for a scalable, distributed system.

\vfill

\selectlanguage{hungarian}
