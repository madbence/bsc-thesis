\section{JavaScript}

A JavaScript egy dinamikus, gyengén típusos nyelv, amit Brendan Eich
fejlesztett ki 1995-ben \emph{Mocha} kódnéven, megjelenésekor LiveScript néven,
és a Netscape böngészőben nyújtott támogatást különböző felhasználói
interakciókhoz.\cite{JSAnnounce}

Szintaktikáját tekintve a C-típusú nyelvek közé sorolható (ahogy a C++, Java és
C\# is), azonban ezektől eltérően az öröklési modellje nem osztály, hanem
prototípus alapú. Ez a különbség sok félreértésre ad okot a nyelv használóinak
körében, megítélése pont emiatt nagyon szélsőséges.

A nyelvet később az Ecma International karolta fel, ami azóta is karbantartja
a nyelv specifikációját.

\subsection{ECMAScript}

Az ECMAScript a JavaScript nyelv specifikációja (a JavaScript név hivatalosan
a Mozilla tulajdona, így valójában a ma elterjedt nyelvre ECMAScript néven
lenne helyes hivatkozni), aminek a sorsáról a TC39 (Technical Committee)
csoport tagjai döntenek.\cite{TC39}

A tagok különböző, főleg IT-val foglalkozó vállalatok által delegált mérnökök,
többek között Google, Intel, Microsoft, Mozilla, Apple, eBay
alkalmazottak.
A nyelv ettől függetlenül a felhasználók aktív közreműködésével készül,
bárki tehet javaslatokat a specifikációt illetően (pl. a specifikáció hivatalos
levelezőlistáján\cite{ESDiscuss}).

\subsubsection{ECMAScript 1}

Az első verzió\cite{ECMAScript-1} 1997-ben jelent meg, gyakorlatilag csak egy
formális specifikációt nyújtott a már meglévő nyelv számára,
így más böngészőkben is azonos működésű nyelvet lehetett implementálni.

\subsubsection{ECMAScript 3}

A második verzió túl sok újdonságot nem tartalmazott, az ECMAScript 3 azonban
számos új nyelvi elemet vezetett be\cite{ECMAScript-3}, többek között
a \verb=try{ } catch(e) { }= szerkezetet, illetve a reguláris kifejezéseket.

A ma használt összes böngésző teljes támogatást nyújt a nyelvhez.

\subsubsection{ECMAScript 4}

A következő verzió olyan sok új elemet tartalmazott, hogy végül sokat közülük
véglegesen elvetettek, másokat a \emph{Harmony} projectbe helyeztek át,
amiből később a 6-os és 7-es verziók merítettek ötleteket.
A specifikáció soha nem készült el teljesen.

\subsubsection{ECMAScript 5}

A 2009-ben elkészült specifikációt a ma használt szinte összes modern böngésző 
támogatja (vagy a javított, 5.1-es változatot\cite{ECMAScript-5}).
A 3-as verzióhoz képest tartalmazza a getterek, setterek specifikációját,
natív JSON támogatást, illetve egységesítette a korábban használt
reflection módszereket.

Tartalmazza az ún. \emph{strict} módot, amivel a leggyakrabban elkövetett
hibákat statikus analízis eszközökkel ki lehet szűrni már a fejlesztés korai
szakaszában.

\subsubsection{ECMAScript 6}

Jelenleg is fejlesztés alatt áll\cite{ECMAScript-6},
jelentős újdonságokat tartalmaz, az egyik legnagyobb újítás az osztályok
és modulok bevezetése, generátorok, beépített \verb=Map=, \verb=Set=,
\verb=WeakMap= adatszerkezetek, a metaprogramozáshoz nélkülözhetetlen
\verb=Proxy=-k.

Jelenleg a böngészők csak részben támogatják, ám folyamatosan zajlik az átállás.
Az általam megírt szoftver erőteljesen épít a generátorokra.

Különböző eszközökkel az ES6 kód ES5-re fordítható (pl. Traceur, 6to5), így
az új verzióban írt kódot is lehet futtatni a mai böngészőkben.
Pl. az AngularJS 2.0 az elsők között lévő keretrendszer,
ami már teljesen ezekre az új alapokra épít.

\subsubsection{ECMAScript 7}

A fejlesztés korai stádiumában van, a tervek között szerepel az \emph{operator
overloading} támogatása, érték típusok (saját primitív típusok), \verb=async=
függvények, mintaillesztés.
