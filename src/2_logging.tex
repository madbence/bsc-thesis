\section{Naplózás}

A naplózást minden rendszer saját hatáskörében végzi el, azonban azokat
egy közös helyre (InfluxDB) is beteszi. Így az adatok központilag elemezhetők,
de lebonthatóak alkalmazásokra, hosztokra, folyamatokra.

\subsection{HTTP kérés naplózás}

A frontend szerver egyszerűen a beérkező HTTP kéréseket naplózza, ezek
válaszidejét méri. Ez a gyakorlatban az OAuth-tal összefüggő kéréseket
jelenti.

\subsection{Adatbázis naplózás}

Az adatbázis külön naplóz minden adathozzáférést, annak típusa szerint
csoportosítva (írás, olvasás), illetve az erőforrás típusa (tweet, user, stb.)
alapján. Ezen felül a HTTP interfészének válaszidejét is méri, így
külön-külön mérhető az adatbázis, illetve a szerver válaszideje
(a szerver válaszideje ekkor az adatbázis és a
HTTP szerver \emph{overhead}jének összege).

\subsection{Háttérfolyamatok}

Az egyes végrehajtó folyamatok is monitorozzák az általuk végrehajtott
feladatokat

\begin{itemize}
  \item Mennyi idő alatt hajtotta végre az adott feladatot
  \item Mekkora volt a Twitter API válaszideje
  \item Mennyi időt töltött a várakozási sorban, amíg sorra került
  \item Hiba esetén naplózni kell a hiba eredetét (adatbázishiba,
    vagy a Twitter hibája)
\end{itemize}

A feldolgozás során a Twitter számos hibát generál. Ezek túlnyomó többsége
a privát Twitter felhasználók miatt van: az ő tweetjeit csak bizonyos
emberek láthatják (akiknek előzetesen jóváhagyta), így ezeket az üzeneteket
az API-ból sem tudjuk elérni. Ilyenkor jobb megoldás híján az adott
\emph{fetch-tweets} jobot befejezettnek tekintjük.

Fel kell készülni az olyan hibákra is, hogy az API (vagy éppen az adatbázis)
elérhetetlen: ilyenkor a jobot nem szabad lezárni, újra be kell ütemezni.
Ez praktikusan egy \verb=job.reject(true)= hívással történik meg, azaz
a feladat visszakerül a feldolgozási sor elejére.
