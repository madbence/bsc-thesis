\section{Twitter API}

A Twitterben rövid szöveges üzeneteket (\emph{tweet}, 140 karakter)
publikálhatunk, követhetünk felhasználókat, akiknek a tweetjei ekkor
megjelennek a saját falunkon.
Lehetőség van direkt egy felhasználó tweetjeinek lekérdezésére,
illetve a követőinek megtekintésére.

Ezeket a funkciókat harmadik fél által írt alkalmazások is képesek elérni,
a Twitter publikus API-ján keresztül.
Az API \emph{restful}, az adatokat kényelmesen JSON formátumban el tudjuk érni.
Az egyetlen hátrány, hogy az API hívások számára igen komoly korlátok
vonatkoznak, negyed órás időablakokban csak adott számú API kérést lehet
elküldeni az egyes végpontokba. Az API kérések egy token birtokában hajthatóak
csak végre.

Az API korlátok az API tokenhez rendeltek, így a limit sok felhasználó
esetén sem jelent szűkebb keresztmetszetet, azonban azok így is
nagyon szigorúak.

Az alkalmazás két végpontot használ:

\begin{description}
  \item[\texttt{GET /followers/ids}] Egy adott felhasználó követőit listázza
    ki JSON tömbként. A limit 15 hívás minden 15 percben, egy hívással
    maximum 5000 követő azonosítója kérhető le. Szerencsére ez nem túl szűk
    korlát, az átlag felhasználó nagyságrendileg nem rendelkezik ekkora
    követő bázissal, de esetenként (pl. egy nagyobb brand esetén) szükség van
    több lekérdezésre.
  \item[\texttt{GET /statuses/user\_timeline}] Egy adott felhasználó legutolsó
    tweetjeit listázza ki JSON formátumban, legfeljebb 3200 tweetre
    visszamenőleg, azonban egy kéréssel maximum 200 tweet kérdezhető le.
    Negyed óránként maximum 180 kérés hajtható végre.
\end{description}

A korlátokból leszűrhető, hogy a követők lekérdezésénél 55000 követő alatt
nem ütközünk problémába (azonnal lekérdezhető), azonban az egyes követők
tweetjeinek megszerzése komoly gondot jelent, negyed óránként maximum
180 követő adatait tudjuk megszerezni. A megoldást erre az jelenti,
ha több tokent is felhasználunk. Bár egy felhasználóhoz csak egy token
tartozik, azok érvényessége nincs limitálva (bár a felhasználó
a Twitter megfelelő felületén visszavonhatja), így azok tetszőlegesen
újrafelhasználhatóak más felhasználók által létrehozott feladatok elvégzésére.

A \verb=/followers/ids= végpont egy JSON objektumot ad vissza,
ami az \verb=ids= kulcs alatt egy tömböt tartalmaz, a követők azonosítóival,
illetve az eredmények közötti navigáláshoz egy \verb=next_cursor= mezőben
a további követők (újabb 5000) lekérdezéséhez szükséges kurzor objektumot
(a végpont opcionálisan elfogad egy \verb=cursor= paramétert).

A \verb=/statuses/user_timeline= egy JSON tömböt ad vissza, amikben a tweet
objektumok vannak. Az alkalmazás számára releváns mezők:

\begin{description}
  \item[\texttt{created\_at}] Az üzenet létrehozásának ideje, UTC idő szerint.
  \item[\texttt{text}] Az esetleges későbbi elemzéshez eltároltam a tweet
    szövegét is.
  \item[\texttt{user.id} és \texttt{user.screen\_name}] A szerző azonosítója.
  \item[\texttt{user.utc\_offset}] A felhasználó időzónája, az esetleges
    további szegmentáláshoz.
\end{description}

Az API-hoz létezik az elérést megkönnyítő Node.JS modul, így az alkalmazásban
azt használom. A \verb=twit= egyszerű hozzáférést biztosít a REST végpontokhoz,
használatához csak a megfelelő tokenekre van szükség, amit az autentikáció
során lehet megszerezni:

\begin{js}
var Twit = require('twit');
var client = new Twit({
  consumer_key: '...',
  consumer_secret: '...',
  access_token: '...',
  access_token_secret: '...'
});

client.get('followers/ids', {
  screen_name: '...'
}, function (err, ids) {
  // ...
});
\end{js}

A modul elrejti előlünk a háttérben történő HTTP kérést, elvégzi a kérés
aláírását (a megadott tokenekkel), szerializálja a metódusnak átadott
paramétereket, végül a kapott választ JSON objektummá konvertálja.
Az alkalmazában ezt a könyvtárait is egy wrapper függvényen keresztül
használom, hogy a generátor-alapú aszinkron kód minél olvashatóbb legyen.
