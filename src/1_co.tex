\section{A \texttt{co} könyvtár}

Az aszinkron működés egyik legnagyobb hátránya, hogy az elkészült program
nehezen átlátható, gyakori az ún. \emph{callback hell} jelenség:

\begin{js}
asyncOp(function() {
  asyncOp(function() {
    asyncOp(function() {
      done();
    });
  });
});
\end{js}

A probléma megkerülésére több megoldás is létezik, a gyakorlatban a
\emph{Promise} illetve a \emph{generátor} alapú megközelítést alkalmazzák.

Az előbbi nem igényel új nyelvi elemet, azonban rendkívül sok \emph{boilerplate}
kóddal jár:

\begin{js}
asyncOp().then(function() {
  return asyncOp();
}).then(function() {
  return asyncOp();
}).then(done);
\end{js}

Pont az ilyen problémák miatt döntöttem amellett, hogy a generátor-alapú
megközelítést használjam az elkészített alkalmazásban.
Ennek használatát könnyíti meg a \verb=co= könyvtár.

\subsection{Generátor függvények}

Az ECMAScript 6 egyik legnagyobb újdonsága a generátorok bevezetése,
amelyek a C\# \verb=yield=\cite{CSYield} kulcsszavához hasonló működést
produkálnak.
Segítségével képesek vagyunk a lusta kiértékelést használva
potenciálisan végtelen szekvenciákon végigiterálni.

\begin{js}
function* squares() {
  for(var i = 1;; i++) {
    yield i * i;
  }
}

for(var n of squares()) {
  if (n > 100) {
    break;
  }
  console.log(n);
}
\end{js}

A példakódban a \verb=function*= kulcsszó jelzi, hogy egy generátor függvényt
definiálunk, a függvény pedig az összes négyzetszámot generálja
(ami természetesen egy végtelen szekvencia). A \verb=for-of= konstrukcióval
a generátor értékein tudunk végigiterálni.
Mivel egy végtelen hosszú listát nem tudunk kiértékelni, az első 10
négyzetszám után megszakítjuk a futást.

Az iterációt megkönnyítő \verb=for-of= csak szintaktikus édesítőszer,
a generátort magunk is léptethetjük a \verb=next()= metódussal:

\begin{js}
var s = squares();
for(;;) {
  var result = s.next();
  if (result.done || result.value > 100) {
    break;
  }
  console.log(result.value);
}
\end{js}

Elsp lépésként példányosítjuk a generátort (szintaktikailag ez megegyezik
a függvényhívással), majd a generátorpéldányt léptetjük a \verb=next()=
metódussal. Ekkor megkapjuk a generátorban a \verb=yield= által kiszámolt
értéket (\verb=result.value=). Ha a generátort nem lehet tovább léptetni,
a \verb=result.done= változóba \verb=true= érték kerül.

\subsection{Aszinkron működés generátorokkal}\label{sec:async}

Sajnos a nyelv a C\#-ban ismert \verb=async-await= konstrukciót nem támogatja
(feltehetően csak az ES7-ben lesz a nyelv része),
azonban a megfelelő \emph{wrapper} függvény meghívásával ahhoz nagyon hasonló
működés emulálható, azaz a megfelelő helyen (aszinkron függvényhívásoknál)
a futás felfüggesztődik, az eseményhurok képes más feladatokat végrehajtani,
majd ha a művelet befejeződött, a vezérlést visszakerül a megfelelő helyre.

Ehhez a már meglévő aszinkron függvényeinket kell becsomagolni úgy,
hogy azok ne vegyenek át \emph{callback} függvényt, hanem egy ún.
\emph{thunk}-ot adjanak vissza, amit aztán a megfelelő modul képes meghívni.

\begin{js}
function thunkify(fn) {
  var args = [].slice.call(arguments);
  var thisArg = this;
  return function thunk(cb) {
    return fn.apply(thisArg, args.concat[cb]);
  };
}
\end{js}

A \verb=thunkify= függvény egy becsomagolandó függvényt (\verb=fn=) vár,
és a becsomagolt függvényt (\verb=thunk=) adja vissza. Ezzel tulajdonképpen
\emph{lekötjük} a függvény első $n$ argumentumát.

A \verb=thunk=-ot generálva a generátort futtató modul egy speciális
callback függvénnyel hívja meg, ami pusztán arra szolgál,
hogy a generátort tovább léptesse.
Ennek a lehető legegyszerűbb megvalósítása:

\begin{js}
function run(gen) {
  var instance = gen();
  function cont(thunk) {
    thunk(function (err, result) {
      var ret;
      if (err) {
        ret = instance.throw(err);
      } else {
        ret = instance.next(result);
      }
      if (ret.value) {
        cont(ret.value);
      }
    });
  }
  cont(instance.next().value);
}
\end{js}

Az általam írt \verb=genzen= könyvtár hasonló funkcionalitást nyújt,
ám nem csak \emph{thunk}-ot fogad el, hanem \emph{promise}-t is,
a szakdolgozatban azonban a \verb=co= modult\cite{Co} használom,
mivel támogatja a párhuzamosítást, illetve mert rengeteg modul érhető el hozzá.

Használata nagyon egyszerű:

\begin{js}
function sleep(ms) {
  return function (cb) {
    setTimeout(cb, ms);
  }
}

co(function* () {
  console.log(1);
  yield sleep(1000);
  console.log(2);
});
\end{js}

A példában a két \verb=console.log= hívás 1000ms különbséggel hívódik meg.

A modul támogatja a párhuzamosságot:

\begin{js}
  co(function* () {
    var pages = yield [
      http.get('https://nodejs.org'),
      http.get('https://npmjs.org'),
    ];
    console.log(pages);
  });
\end{js}

A példában a \verb=pages= változó értékébe a felsorolt két honlap tartalma
kerül, a letöltés párhuzamosan történik, a művelet végén a szinkronizációt
is megoldja a modul.
