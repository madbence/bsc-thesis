\section{Monolitikus architektúra}

Egy \emph{monolitikus} alkalmazás önálló egységet alkot, kompakt, viszonylag
kevés külső függőséggel rendelkezik. Többnyire rétegelt, vagy hexagonális
felépítéssel bír. Ez több előnnyel is jár:

\begin{itemize}
  \item A fejlesztés egyszerű, hiszen a rendelkezésre álló fejlesztőkörnyezetek
    (pl. Visual Studio) ezt a fajta architektúrát támogatják.
  \item Egyszerű az üzemeltetés, hiszen az alkalmazást könnyű deployolni, mivel
    az egyedül is képes a komplett funkcionalitást biztosítani.
  \item Több példányt elindítva a skálázás megoldható, akár automatikusan is,
    ehhez a gyakorlatban csak egy terheléselosztóra van szükség.
\end{itemize}

A hátrányai pont ezek miatt:

\begin{itemize}
  \item A komplex alkalmazások komplex fejlesztőkörnyezetet igényelnek, ami
    megnövekedett erőforrásigénnyel jár.
  \item A komplexitás következtében az alkalmazás nagyon nagy, adott esetben
    az inicializálás ideje nagyon nagyra nőhet, ami a skálázhatóságot rontja.
  \item A skálázási lehetőségek nagyon limitáltak, legtöbb esetben
    erőforráspazarlóak, hiszen az egyetlen lehetőség a több példányos futtatás.
    Ezzel nem csak a szűk keresztmetszetet jelentő komponenst duplikáltuk,
    hanem mindent.
  \item A szoros csatolás miatt megnő a \emph{vendor lock-in} veszélye,
    azaz más technológiára váltás, vagy akár csak egy verziófrissítés
    nagyon komoly kihívást jelent.
  \item Bár egy szakdolgozatban nem okoz problémát, de valódi enterprise
    környezetben egy folyamatban lévő projectbe bekapcsolódás nagyon nehézkes,
    hiszen a fejlesztőnek meg kell értenie az alkalmazást, át kell látnia azt.
\end{itemize}
