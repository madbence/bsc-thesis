\subsection{Sablonkezelés}

Az alkalmazásoknál gyakori, hogy az egyes DOM elemek dinamikusan generálódnak,
ekkor szükség van valamilyen \emph{template} fájlra, amibe különféle változókat
behelyettesítve megkapjuk a végleges HTML kódot.

Erre a célra a Jade könyvtárat\cite{Jade} használtam,
ami HTML kód generálására képes, és nagyon elegáns szintaxissal rendelkezik:

\begin{jade}
doctype html
html
  head
    title Teszt
  body
    h1 Hello world
    a(href='http://nodejs.org') Ez egy link
\end{jade}

Aminek az eredménye fordítás után:

\begin{html}
<!DOCTYPE html>
<html>
  <head>
    <title>Teszt</title>
  </head>
  <body>
    <h1>Hello world</h1><a href="http://nodejs.org">Ez egy link</a>
  </body>
</html>
\end{html}

Természetesen vezérlési szerkezetek is használhatók:

\begin{jade}
h1 Teendők:
ul
  each item in list
    li
      span!= item.name
      if item.done
        span (kész)
\end{jade}

Az elkészült \verb=.jade= kiterjesztésű fájlok kliens oldalon egyszerűen
felhasználhatóak a \verb=jadeify= segítségével: a kívánt sablon egy egyszerű
\verb=require= hívással betölthető, \verb=browserify= által lefordított
fájlba automatikusan bekerül a sablon optimalizált, végrehajtható változata:

\begin{js}
var template = require('template/todo.jade');
var todoList = [{
  name: 'Foo',
}, {
  name: 'Bar',
  done: true,
}];

document.body.innerHTML = template(todoList);
\end{js}

A jade sablonnyelv adatkötési réteget nem biztosít, ha erre is szükség van,
akkor érdemes külső megoldást használni, ilyen a Facebook által fejlesztett
React (létezik jade integráció is, a \verb=react-jade= csomagot
kell telepíteni).

A kliens oldali logika nem indokolta a bonyolult adatkötéseket, így az
elkészült szoftverben csak statikus jade sablonokat használtam.
