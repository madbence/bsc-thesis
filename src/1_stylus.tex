\section{Stylus}

A böngészőben történő korszerű megjelenéshez szükség van CSS stílusfájlokra,
azonban a leírónyelv meglehetősen kényelmetlen, így helyette a CSS-re forduló
Stylus\cite{Stylus} nyelvet használtam.

A szintaxis támogatja az egymásba ágyazott CSS szelektorokat, illetve
számos programozási nyelvi eszköz érhető el benne
(változók, ciklusok, elágazások, függvények, stb.),
de a legnagyobb erénye, hogy rendkívül kompakt:

\begin{stylus}
body
  font-size 16pt
  h1
    color red
\end{stylus}

A fenti kódból a \verb=stylus= bináris az alábbi CSS kódot generálja:

\begin{css}
body {
  font-size: 16pt;
}
body h1 {
  color: #f00;
}
\end{css}

Maga a fordító szintén Node.JS alapú, az egyszerű fordításon kívül
természetesen a fejlesztést nagyban megkönnyítő automatikus fordítást is
támogatja:

\begin{sh}
  $ stylus --watch style.styl -o style.css
\end{sh}
