\section{Adatbázis szerver}

Az adatbázis szerver feladata az alkalmazás állapotának számon tartása.
Ezek közé tartozik az egyes \emph{business objektumok} perzisztálása,
illetve ezek menedzselése egy egyszerű REST HTTP JSON interfészen keresztül.

Az üzleti logika szempontjából releváns entitások a következők:

\begin{description}
  \item[user] a felhasználót reprezentáló objektum, a hozzá tartozó
    legfontosabb információkkal (regisztráció időpontja, e-mail cím, stb),
    illetve az elkészült statisztikákat is ebben az objektumban tárolom el.
  \item[token] az egyes Twitter API hívásokhoz szükség van a autentikációs
  tokenekre, ezeket a felhasználótól kapja meg az alkalmazás, az OAuth
  autentikációs során. Fontos számon tartani az állapotát, hányszor lett
  felhasználva, mikor lett utoljára felhasználva.
  \item[tweet] a Twitter API válaszaiból leképzett objektumok, az alkalmazás
  szempontjából az egyetlen releváns adata a tweet időpontja, hiszen ebből
  határozza meg a szoftver az ideális időpontot a tartalom publikálására.
  \item[job] az egyes elvégzendő háttérfeladatok. Az alkalmazás eltárolja a
  a teljes életciklusát, tehát hogy mikor hozták létre, mikor került egy
  feldolgozó egységhez, mikor végzett az vele. Természetesen a job állapota
  is tárolva van, ezzel a többszörös végrehajtást lehet kivédeni.
\end{description}

Egy felhasználó létrehozásakor vele együtt létrejön egy \emph{fetch-followers}
job is, ez indítja be az adott felhasználóhoz tartozó statisztikák
összegyűjtését.

Egy job létrehozásakor nem csak egy adatbázisrekord jön létre, hanem a
hozzá tartozó üzenet bekerül a várakozási sorba (ez az üzenet egyszerűen
a job azonosítóját tartalmazza), amit majd az épp aktív feldolgozóegységek
végrehajtanak.

A felhasználókhoz el van tárolva az aktív, illetve függőben lévő jobok
száma. Mivel ezeket az értékeket inkrementálni kell, amit a LevelDB atomi
műveletként nem támogat, így ezt nekem kellett megvalósítanom.
Ehhez írtam meg a \verb=co-lock= modult, ami kölcsönös kizárást biztosít
erőforrásokhoz, használata pedig a már megszokott \verb=yield= szintaktikával
lehetséges:

\begin{js}
  var lock = require('co-lock');

  var release = yield lock(resource);
  // kritikus szakasz
  yield release;
\end{js}

A háttérben a modul a \verb=resource= erőforráshoz egy zárat hoz létre,
vagy ha az már zárolva van, akkor annak a \emph{várakozási sorába} lép be,
a \verb=yield release= ezt a zárat engedi el, azaz a várakozási sorban
következő hívás hajtódik végre.

Zárak használatakor ügyelni kell a \emph{deadlock} elkerülésére.
Szerencsére az alkalmazásban máshol nem volt szükség zárakra, ezért
a deadlock nem fordulhat elő (hiszen ahhoz több erőforrásra van szükség,
az adatbázis azonban műveletenként csak egyet használ).
