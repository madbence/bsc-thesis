\section{\texttt{browserify}}

A nagyobb méretű webes kliensoldali alkalmazások egyik nagy problémája,
hogy a JavaScript nyelvi szinten nem ismeri a \emph{modul} fogalmát
(bár jelenleg az ES6 részeként már kidolgozás alatt áll a
specifikáció\cite{ES6-Modules}),
így komplex szoftverek csak rendkívül kényelmetlenül írhatók benne.

A problémára több megoldás is született, mindkét megközelítés a modul fogalmát
ragadja meg, ezek közül az egyik a CommonJS, a másik pedig az AMD
(Asynchronous Module Definition) specifikáció.
Mindkettőre létezik több implementáció is (AMD API-t biztosít pl. a RequireJS).

Mivel Node.JS környezetben a CommonJS a megszokott modul-implementáció, így
kézenfekvő, hogy az alkalmazás kliens-oldali részéhez is érdemes ezt használni,
ennek számos előnye van : pl. kódmegosztás, illetve maga a tény, hogy szinte
ugyanaz a rendszer van mind kliens mind szerver oldalon (így az npm-ben lévő
több tízezer csomag képes a böngészőben is futni).

Az egyik legnépszerűbb eszköz a CommonJS modulok \emph{becsomagolására}
a \verb=browserify=\cite{Browserify},
ami képes a program minden függőségét egyetlen JavaScript fájlba csomagolni:

\begin{sh}
$ browserify app.js -o bundle.js
\end{sh}

Ahol az \verb=app.js= az alkalmazás belépési pontja, a \verb=bundle.js=
pedig a kimeneti fájl. A program felderíti az alkalmazás függőségeit,
azokat a megfelelő wrapper függvényekbe csomagolja, majd konkatenálja.

\subsection{Inkrementális fordítás}

Mivel a fordítási folyamat adott esetben néhány másodpercig is tarthat,
így érdemes a \verb=watchify= csomagot\cite{Watchify}
használni, ami képes az inkrementális fordításra,
illetve a fájlok változása esetén automatikusan újrafordítja
az alkalmazást, gyakorlatilag néhány milliszekundum alatt:

\begin{sh}
$ watchify app.js -o bundle.js
\end{sh}

A fordítási folyamat egyszerűen bővíthető pluginekkel, így a lefordított
fájlhoz akár \emph{source map} is rendelhető, amivel a fordítás miatt
keletkező debugolási problémák (nem lehet visszakövetni, hogy a lefordított,
és akár minifikált fájlban az egyes szimbólumok hol vannak a forrásfájlokban)
gyakorlatilag megszűnnek.
