\subsection{\texttt{logify}}

Mivel a cél egy lazán csatolt rendszer megtervezése, így nem célszerű
közvetlenül az InfluxDB interfészét használni, azt érdemes egy köztes rétegen
keresztül használni. Ennek a szerepét tölti be az általam írt \texttt{logify}
modul.\cite{Logify}

A modul egyszerű metódusokat biztosít események naplózására, amik a háttérben
tetszőleges (akár egyszerre több) rendszernek továbbítódnak.

\begin{js}
var logify = require('logify');
var logger = logify();

logger.info('This is a message %d', 1337);
logger.debug({
  foo: 'bar',
}, 'This is a debug message with context');
\end{js}

Az elérhető naplózási szintek megfelelek a sztenderd \emph{syslog} szinteknek,
így alapértelmezetten a \verb=logify-syslog-levels= csomagban definiált
szinteket használja, de ez természetesen felüldefiniálható.

A naplózó rendszer alapértelmezetten a sztenderd kimenetre ír, ám tetszőleges
másik \emph{transport} is hozzáadható, többek között InfluxDB csatoló is.

Az egyes események rövid szöveges üzenettel jellemezhetőek, azonban csatolható
hozzájuk kontextus, pl. az egyes metrikák aktuális értékei, az esemény típusa,
stb. Az egyes transportok saját hatáskörükben döntik el, hogy ezekkel az
értékekkel mit kezdenek, hogyan szerializálják, illetve hogyan továbbítják
a távoli szervernek.

A \verb=child(ctx)= metódussal a naplózó egy kiterjesztett példányát lehet
létrehozni, ami minden eseményhez automatikusan csatolja a paraméterben
átadott \verb=ctx= kontextus-objektumot. Ezzel a módszerrel könnyen tudjuk
fa struktúrába szervezni a naplózó rendszereinket.
Az alkalmazásban azt a megközelítést használom, hogy minden HTTP kérés
saját naplózót kap, amihez a HTTP kérés kontextusa automatikusan csatolva
van, így minden eseménynél tudjuk, hogy azt pontosan milyen HTTP kérés
váltotta ki.
