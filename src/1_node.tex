\section{Node.js}

A Node.js a Google által fejlesztett \emph{v8} JavaScript értelmezőt használó
JavaScript futtatókörnyezet\cite{Node}, ami a Ruby-ból ismert Event Machine
\cite{EventMachine} és a Python Twisted\cite{Twisted} moduljához hasonlóan
működik, ám a felhasználó elől teljesen elrejti az \emph{eseményhurkot}
(\secref{async-io}. fejezet).

Az első verziót Ryan Dahl írta 2009-ben, azóta a projekt vezetése TJ Fontaine
kezébe került\cite{Node-TJ}, jelenleg is fejlesztés alatt áll, a következő stabil verziója
a v0.12 lesz. A szakdolgozatban elkészített szoftver a node 0.11-es verzióján
fut, mivel sok olyan funkciót használ, ami a stabil ágon (0.10) nem elérhető.

A Node.js-t már számos nagyvállalat használja, ami képes a régi,
rugalmatlan Java (vagy .NET) alapú stack kiváltására.
Az egyik legnagyobb sikertörténet a PayPalhoz fűződik, ahol a frontend
szoftvert (a PayPal webes felületét) újraírták Java és Node.js alapokon is.
\cite{PayPal-Case}

A kísérlet során két ugyanolyan funkcionalitással (azonos specifikációval)
rendelkező szoftver implementáltak Java-ban és Node-ban. Az eredény:
az utóbbit fele annyi mérnök fele annyi idő alatt fejezte be,
a kész szoftver pedig kétszer annyi HTTP kérést bírt kiszolgálni
másodpercenként, mint az ekvivalens Java alkalmazás
(a válaszidő is javult a harmadával), a Node.js verzió pedig nem is
használta ki a Java verzióval ellentétben a többmagos környezet előnyeit.

\subsection{Aszinkron IO}\label{sec:async-io}

A Node.js a grafikus rendszerekben megszokott \emph{eseményhurok}-ra épít,
azaz egy végtelen ciklusban várakozik eseményekre, majd azokat a megfelelő
\emph{callback} függvényeknek továbbítja.

A platformon alapvetően minden IO-t érintő hívás aszinkron,
azaz a függvényhívás nem blokkol, hanem az IO művelet befejezése után
a megadott \emph{callback} függvény fut le. Természetesen lehetőség van
szinkron végrehajtásra is, azonban ezeket egyszerű konzolos
alkalmazásokhoz érdemes csak használni.

Szemléltetésképpen egy szinkron hívás:

\begin{js}
var result = http.get('http://nodejs.org');
\end{js}

A \verb=http.get= függvény itt blokkol, azaz addig, amíg a művelet nem
fejeződik be, a program nem tud tovább futni, ``lefagy''.
Ezt elkerülendő, a Node az aszinkron működést támogatja:

\begin{js}
http.get('http://nodejs.org', function(err, result) {
  // hiba, illetve az eredmény kezelése
});
\end{js}

A \verb=http.get= itt rögtön visszatér (végül pedig az eseményhurokban fog
várakozni az alkalmazás), és ha megérkezett a válasz(esemény),
a futtatókörnyezet lefuttatja a hozzárendelt \emph{callback} függvényt.

Hagyományos szinkron hívásoknál ha több \verb=get= hívást szeretnénk
végrehajtani, akkor az egyetlen lehetőségünk a szálak használata.
A szálak nem skálázhatóak, erőforrás-pazarlóak (rengeteg memóriát igényelnek,
mivel saját \emph{stack}-re van szükségük).
Egy node folyamat képes akár párhuzamosan több tízezer HTTP kérést
is kiszolgálni, a hagyományos, thread pool alapú megoldással ez
elképzelhetetlen.
Volt már rá példa, hogy egyetlen node folyamat egymillió TCP kapcsolatot
tartott fönn\cite{Node-1M}, mindössze 16GB memóriát igényelve.
Szál-alapú megoldással ugyanehhez 4TB memóriára volna szükség (amennyiben
az átlagosnak mondható 4MB-os stack mérettel számolunk)

A megoldás hátránya a bonyolultabb kód, erre a problémára a generátorok
jelentik a megoldást (\secref{async}. fejezet).

Az egyszálú működés másik hátránya, hogy a különösen CPU-intenzív feladatokra
nem célszerű használni, hiszen bár az IO hívások aszinkronak, a többi
művelet szinkron, így egy sokáig blokkoló (CPU-t használó) kérés miatt az
utána érkező kérések addig nem kerülnek kiszolgálásra, amíg az előző be nem
fejeződött, nem tapasztaljuk a szálaknál megszokott preemptív működést.

A problémára részben megoldást nyújthat, ha több node kiszolgáló folyamatot
indítunk el (pl amennyi processzor rendelkezésre áll a rendszeren), ezzel
az egyszálú működés hatékonyságán javítottunk, azonban a node nem-preemptív
természetén nem változtat.

Természetesen létezik megoldás arra, hogy szálakat használjunk: natív modulokat
kell írnunk (C++-ban), vagy egy erre specializált modult használnunk,
amivel JavaScriptet futtató szálakat tudunk létrehozni, ez utóbbira példa
a \verb=threads_a_gogo= könyvtár.\cite{Threads-a-gogo}
Jelen szakdolgozatnak nem célja ezeket a
megoldásokat bemutatni, így a továbbiakban mindig egyszálú működést lehet
feltételezni.

\subsection{CommonJS modulok}

A node programok modulokból épülnek fel, amik a CommonJS specifikációt követik
\cite{CommonJS}.
Minden modul a következő struktúrát mutatja:

\begin{js}
var beep = require('beep');

module.exports = function beepboop() {
  beep();
  console.log('boop');
};
\end{js}

Azaz meglévő modulokat a \verb=require= függvénnyel tudunk beimportálni,
a modul által nyújtott API-t pedig a \verb=module.exports=
(illetve \verb=exports=) változón keresztül tudjuk módosítani.

\subsection{npm}

A node szerves része a beépített csomagkezelő, az \verb=npm=, amivel a
felhasználók által írt modulokat tudjuk telepíteni, illetve magunk is
publikálhatunk modulokat. Jelen sorok írásakor az \verb=npm= csomagok száma
már átlépte a 100000-t\cite{NPM-100K},
ez több, mint bármelyik másik csomagkezelő rendszerben lévő csomagok száma
(a második legnagyobb a Java Maven, körülbelül 80000 csomaggal).

Egy csomag az

\begin{sh}
$ npm install [név]
\end{sh}

paranccsal telepíthető.

A csomagok része a \verb=package.json= fájl, ami metainformációkat tartalmaz a
csomagról: név, verzió, függőségek, a részletes leírás megtalálható az
\verb=npm= oldalán\cite{Package-JSON}.

\subsection{Szemantikus verziókezelés}

Az \verb=npm= modulok az ún. \emph{semantic versioning}-et követik\cite{Semver},
ez azt jelenti, hogy a modul három verziószámmal rendelkezik:

\begin{description}
\item[major] Fő verzió szám, az azonos számmal rendelkező release-ek garantáltan
azonos API-val rendelkeznek, így a csomagok frissítése egyszerűen
automatizálható
\item[minor] Alverzió, a visszafelé kompatibilis változásokat lehet ennek a
számnak a növelésével jelezni.
\item[patch] A kisebb hibajavítások után célszerű ezt a számot növelni.
\end{description}

Amennyiben a publikált modulok betartják ezeket a konvenciókat, úgy
a csomagfrissítések automatizálhatóak, hiszen a modul bármikor magasabb
\emph{minor} verzióra frissíthető, mert az garantáltan visszafelé kompatibilis.
